\documentclass{article}
\usepackage{hyperref}
\usepackage{graphicx}
\graphicspath{ {pictures/} }

\title{MaMEA User's Guide}
\author{Lukas Turcani}
\begin{document}

\begin{titlepage}
	\maketitle
\end{titlepage}

\tableofcontents
\newpage
\section{Introduction}
MaMEA (MacroMolecular Evolutionary Algorithm) is a genetic algorithm (GA) for chemistry. It aims to be as general as possible, being suitable for use with any class of molecules. While it does not support all types of molecules out of the box it is designed to be easily extended. This is made easier by being written in Python. For notes on how to extend MaMEA to a new class of molecules or add GA operations, see the developer's guide.

\section{Installing MaMEA}

\section{Input Files}

To run MaMEA you need to submit an input file. These contain all the details required to start a GA calculation. Among other things, this may include things like which the fitness function you want to use, the number of generations MaMEA should make and how big the population should be.

Using an example, the format and sturcture of input files is explained in section \ref{input_files_format_and_structure}. A guide to finding out what GA operations and tools are available is in section \ref{input_files_valid_parameters}.

\subsection{Format and structure.}
\label{input_files_format_and_structure}

An example input file can be downloaded from \href{}.


\subsection{Valid parameters.}
\label{input_files_valid_parameters}

\section{Running MaMEA}

\subsection{From the command line.}


\section{Output}

\section{MaMEA as a Library}

\end{document}